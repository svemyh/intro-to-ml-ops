\documentclass[11pt]{beamer}
\usetheme{Madrid}
\usefonttheme{serif}

\usepackage[utf8]{inputenc}
\usepackage[english]{babel}
\usepackage[T1]{fontenc}

\usepackage{amsmath}
\usepackage{amsfonts}
\usepackage{amssymb}
\usepackage{graphicx}

\DeclareMathOperator{\sen}{sen}
\DeclareMathOperator{\tg}{tg}

\setbeamertemplate{caption}[numbered]

\author[Sveinung Myhre]{Sveinung Myhre}
\title{Intro to ML-ops}
% Email contact
\newcommand{\email}{sveinung.myhre@example.com}
%\setbeamercovered{transparent}
\setbeamertemplate{navigation symbols}{}
%\logo{\includegraphics[scale=0.08]{imagens/logomarca_profmat.png}}
\institute[]{ReLU NTNU}
\date{September 30, 2025} 
%\subject{}

% ---------------------------------------------------------
% Selecione um estilo de referência
\bibliographystyle{apalike}

%\bibliographystyle{abbrv}
%\setbeamertemplate{bibliography item}{\insertbiblabel}
% ---------------------------------------------------------

% ---------------------------------------------------------
% Incluir os slides nos quais as referências foram citadas
%\usepackage[brazilian,hyperpageref]{backref}

%\renewcommand{\backrefpagesname}{Citado na(s) página(s):~}
%\renewcommand{\backref}{}
%\renewcommand*{\backrefalt}[4]{
%	\ifcase #1 %
%		Nenhuma citação no texto.%
%	\or
%		Citado na página #2.%
%	\else
%		Citado #1 vezes nas páginas #2.%
%	\fi}%
% ---------------------------------------------------------

\begin{document}

\begin{frame}
\titlepage
\end{frame}

\begin{frame}{Outline}
\tableofcontents
\end{frame}

\section{Introduction}

\begin{frame}{Introduction}
Welcome to this brief introduction to ML-ops. Today we will cover:

\begin{itemize}
    \item Docker containers
    \item Kubernetes
    \item GPU glossary
    \item Hands-on coding and pair-programming
\end{itemize}
\end{frame}

\section{Docker Containers}

\begin{frame}{Docker Containers}
    \begin{block}{What are Docker Containers?}
    Lightweight, portable environments that package applications with their dependencies.
    \end{block}

    \begin{block}{Key Benefits}
    \begin{itemize}
        \item Consistency across environments
        \item Resource efficiency
        \item Easy deployment and scaling
    \end{itemize}
    \end{block}
\end{frame}

\section{Kubernetes}

\begin{frame}{Kubernetes}
    \begin{block}{Container Orchestration}
    Kubernetes manages containerized applications across clusters of machines.
    \end{block}

    \begin{block}{Key Features}
    \begin{itemize}
        \item Automatic scaling
        \item Load balancing
        \item Self-healing
        \item Rolling updates
    \end{itemize}
    \end{block}
\end{frame}

\section{GPU Glossary}

\begin{frame}{GPU Terminology for ML}
    \begin{itemize}
        \item \textbf{CUDA}: Parallel computing platform for NVIDIA GPUs
        \item \textbf{Tensor Cores}: Specialized units for AI workloads
        \item \textbf{VRAM}: Video memory for storing model parameters
        \item \textbf{FP16/FP32}: Floating-point precision levels
        \item \textbf{Multi-GPU}: Using multiple GPUs for training
    \end{itemize}
\end{frame}

\section{Hands-on Coding}

\begin{frame}{Pair Programming Session}
    Time for hands-on practice!

    \begin{itemize}
        \item Setting up ML containers
        \item Deploying with Kubernetes
        \item GPU optimization
        \item Best practices discussion
    \end{itemize}
\end{frame}

\section{Questions \& Discussion}

\begin{frame}

\begin{center}
    Thank you!

    Questions?

    \bigskip

    \email
\end{center}

\end{frame}

\end{document}